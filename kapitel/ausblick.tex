\chapter{Ausblick}

\section{Visualisierung von Wetter- und Wellendaten}

\section{Verbesserung der Vorhersagen}
Die aus dem \textit{Global Forecast System} und dem \textit{Wave Watch
  III} Modell stammenden Wetter- und Wellenvorhersagen beruhen zwar
auf einem numerisch berechneten Wettermodell, können aber nur bedingt
zur Vorhersage der tatsächlichen Surfbedingungen verwendet werden. In
Kombination mit Erfahrungswerten und der Kenntnis lokaler
Gegebenheiten dienen sie eher als Indikator zur Einschätzung dieser
Bedingungen. In welcher Verbindung die Vorhersagewerte mit der
Qualität der vor Ort brechenden Wellen stehen ist unklar und variiert
wohl von Spot zu Spot. Zudem kommt, dass wegen der groben
Gitterauflösung der Modelle die Vorhersagewerte teilweise aus der
näheren Umgebung eines Spots herangezogen werden müssen. Trotzdem sind
aber vor allem die berechnete Wellenhöhe, die Wellenperiode, die
Windstärke und die Windrichtung wichtige Indizien, die bei der
Einschätzung der Surfbedingungen eine große Rolle spielen. Was noch
fehlt ist eine Komponente mit der die lokalen Gegebenheiten eines Surf
Spots mit ins Spiel gebracht werden. Außerdem wäre es wünschenswert
eine Aussage über die Qualität der Surfbedingungen an einem Spot
treffen zu können.

\subsubsection{Beobachtung lokaler Gegebenheiten}
Würde man die Qualität der Surfbedingungen an einem Spot mit den dort
prognostizierten Wetter- und Wellenvorhersagen über einen längeren
Zeitraum hinweg vergleichen, könnte man wahrscheinlich bestimmte
Zusammenhänge feststellen. Beispielsweise ist es durchaus üblich, dass
zwei benachbarte Spots sehr ähnlichen Wetter- und Wellenverhältnissen
ausgesetzt sind, die Qualität der brechenden Wellen sich aber
erheblich unterscheidet. Dies kann z.B. daran liegen, dass der eine
Spot besser bei Flut und der andere besser bei Ebbe bricht. Manche
Spots fangen erst bei einer bestimmten Wellenhöhe an zu brechen,
andere hingegen sind bei zu großen Wellen nicht mehr surfbar, da die
Wellen sofort in sich zusammenfallen. Vor Ort ansässige Surfer,
sogenannte \textit{Locals}, haben oft ein feines Gespühr für diese
Zusammenhänge, was wohl auch auf deren langjährige Beobachtungen
zurückzuführen ist. 

Diese Beobachtungen von den Benutzern der Web Applikation vornehmen zu
lassen ist eher unrealistisch. Zum einen sind solche Beobachtungen
sehr aufwendig, und zum anderen müsste sichergestellt sein, dass diese
diszipliniert und nach den selben Kriterien durchgeführt werden. Zudem
stellt sich die Frage wie diese Beobachtungen maschinell verwertet und
in Verbindung mit den Vorhersagen gebracht werden können.

\subsubsection{Abstimmung durch die Benutzer}
Eine im Rahmen dieser Arbeit leider nicht weiter verfolgbare Idee,
welche diese Problematik vielleicht lösen könnte, wäre die Einführung
eines Abstimmungssystems in Kombination mit Data Mining
Verfahren. Informiert sich ein Benutzer auf der Webseite über die
Surfbedingungen an einem bestimmten Spot, könnte man ihn zu den
Bedingungen in den letzten Tagen befragen. Mit etwas Glück kann es
nämlich durchaus sein, dass der Benutzer dort surfen war und eine
Aussage über die damalige Qualität der Wellen treffen kann. Diese
Befragung sollte benutzerfreundlich und schnell von statten gehen,
damit sich die Besucher nicht belästigt fühlen. Hier würde sich
z.B. eine Bewertungsskala von 1 bis 5 ''Sternen'' mit einer Auswahl
des entsprechenden Zeitpunktes anbieten. Die so erhobenen Datensätze
sollten Informationen über den Spot, den Zeitpunkt, die Qualität der
Wellen und den Benutzer (optional) enthalten. Sammelt man viele dieser
Bewertungen kann eine Aussage darüber getroffen werden, wie gut die
Surfbedingungen in der Vergangenheit waren. Außerdem könnte man mit
diesen Informationen z.B. die besten Spots in einer Region bestimmen
oder die Konsistenz der Surfbedingungen an einem Spot ermitteln. Eine
viel interessantere Verwendung dieser Daten wäre aber, sie für die
Bewertung von zukünftigen Surfbedingungen einzusetzen.

\subsubsection{Bewertung zukünftiger Surfbedingungen}
Es wäre wünschenswert, wenn man einem Benutzer zu den Wetter- und
Wellenvorhersagen auch noch eine Einschätzung zur Qualität der
Surfbedingungen geben könnte, in der die lokalen Gegebenheiten eines
Spots mit berücksichtigt sind. Diese Einschätzung sollte dem Benutzer
in der ihm bekannten Bewertungsskala aus dem Abstimmungssystem
präsentiert werden. Je besser die Surfbedingungen, desto mehr
''Sterne''. Der einer Bewertung zugrunde liegende Ansatz besteht in
der Vermutung, dass ähnliche Wetter- und Wellenvorhersagen auch
ähnliche Surfbedingungen provozieren. Wurden für einen Zeitpunkt in
der Vergangenheit positive Bewertung abgegeben, sind die
Surfbedingungen zu einem zukünftigen Zeitpunkt mit ähnlichen Wetter-
und Wellenverhältnissen vielleicht auch gut. Durch den Vergleich der
historischen Vorhersagedaten mit denen der Zukunft, und einer
Strategie aus den abgegebenen Bewertungen welche für die Zukunft
vorherzusagen, könnte man in einigen Fällen vielleicht sogar
vernünftige Ergebnissen erzielen.

\subsubsection{Bewertungen als Klassifikationsproblem}
Die hier zu lösende Aufgabe ist ein sogenanntes Klassifikationsproblem
bei dem die Objekte einer Menge, deren Klassenzugehörigkeit nicht
bekannt ist, einer bestimmten Klasse zugeordnet werden müssen. Als
Klassifikator bezeichnet man den Algorithmus oder das Verfahren nach
dem diese Zuordnung durchgeführt wird. Klassen würden hier durch die
in mehrere Abschnitte unterteilte Bewertungsskala repräsentiert,
Objekte durch die Vorhersagen der Spots. Die Bewertung der
Surfbedingungen ist somit davon abhängig, welcher Klasse eine
Vorhersage zugeordnet wird. Wüsste man für alle Spots wie sich die
Variablen einer Vorhersage auf die Qualität der Wellen auswirken,
könnte man für jeden Spot einen Klassifikator konstruieren, der
idealerweise auch noch die lokalen Gegebenheiten mit einbezieht. Die
Vorschriften wie solch ein Klassifikator jedoch konstruiert wird sind
jedoch meistens nicht bekannt, bzw. nur mit sehr hohem Aufwand zu
ermitteln.

\subsubsection{Anwendung von Data Mining Verfahren}
Unter \textit{Data Mining} versteht man die Anwendung verschiedenster
Algorithmen und Verfahren auf einem Datenbestand, mit dem Ziel meist
noch nicht bekannte Muster zu entdecken. In \cite{Seagaran2007} ist
nicht nur eine Zusammenfassung der in diesem Gebiet häufig verwendeten
Algorithmen zu finden, sondern auch eine Diskussion über deren Vor-
und Nachteile bei der Anwendung auf repräsentative Probleme. Einige
der dort vorgestellten Algorithmen könnten dazu verwendet werden
Zusammenhänge zwischen den Vorhersagen, den lokalen Gegebenheiten und
der Qualität der Surfbedingungen zu finden und daraus zukünftige
Bewertungen abzuleiten. Der unter dem Namen \textit{Support Vector
  Machine} bekannte Algorithmus scheint ein viel versprechender
Kandidat zu sein, um das hier beschriebene Klassifikationsproblem zu
lösen. Zum einen ist der Algorithmus bei der Klassifikation von
Objekten im Vergleich zu anderen Algorithmen sehr schnell, und zum
anderen arbeitet er auch bei einer hohen Anzahl von Dimensionen recht
zuverlässig. Mit Dimensionen sind bei dem hier betrachteten Problem
die Vorhersageelemente der beiden Wettermodelle gemeint. Damit man den
Algorithmus überhaupt einsetzen kann benötigt man sogenannte
Trainingsdaten. Diese Daten werden bei der Initialisierung des
Algorithmus analysiert und spielen bei einer späteren Klassifikation
der Objekte eine Rolle. Für jeden Spot würde man hier eine Instanz des
Algorithmus initialisieren, und diese Instanz mit den historischen
Vorhersagedaten und den durchschnittliche Bewertungen des Spots
trainieren. Werden neue Vorhersagedaten geladen, könnte man die
Vorhersagedaten der jeweiligen Instanz zur Berechnung einer Bewertung
übergeben.

\subsubsection{Erfolgsaussichten und Hindernisse}
Damit mit diesem Algorithmus jedoch gute Ergebnisse erzielt werden
können benötigt man eine größere Anzahl an Trainingsdaten. Da diese
zum jetzigen Zeitpunkt nicht vorhanden sind, kann dieser Ansatz leider
nicht weiter verfolgt werden. Ob dieses Verfahren überhaupt zum Ziel
führt müsste durch eine Auswertung über einen längeren Zeitraum an
verschiedenen Spots überprüft werden. Falls die Ergebnisse einer
solchen Auswertung positiv sind, hätte man einen Weg gefunden die
lokalen Gegebenheiten an einem Spot über die Abstimmungsdaten indirekt
in die Vorhersagen mit einfließen zu lassen.

%%% Local Variables:
%%% mode: latex
%%% TeX-master: "../community-plattform"
%%% End:
