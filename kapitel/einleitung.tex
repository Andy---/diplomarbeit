
\section{Einleitung}

\subsection{Motivation}

Mit Surfen oder Wellenreiten bezeichnet man eine Wassersportart, bei
der versucht wird, stehend eine brechende Welle mit Hilfe eines
Surfbretts entlang zu fahren. Anders als beim Windsurfen wird hier
nicht die Kraft des Windes, sondern die Kraft der brechenden Welle
benutzt, um die zum Aufstehen und Fahren benötigte Geschwindigkeit zu
erreichen. Zum Surfen geeignete Wellen brechen idealerweise
kontinuierlich in eine Richtung.

Diese Wellen sind allerdings nicht immer dort anzutreffen wo es ein
Meer oder einen Strand gibt. Vielmehr sind für Surfer interessante
Wellen an den Orten zu finden, im folgenden \textit{Spots} genannt,
deren geografische Lage die Entstehung von surfbaren Wellen
begünstigt.

Die Beschaffenheit des Untergrunds, über dem die Wellen brechen, ist
dabei einer der wichtigsten Faktoren. Die bei Surfern sehr beliebten
\textit{Pointbreaks} brechen meistens an einer bestimmten Stelle über
festem Untergrund. Diese sind die beständigsten, am besten
einschätzbaren, aber auch gefährlichsten Spots, denn hier brechen die
Wellen meistens über Riffen oder steinigem Boden. Besteht der
Untergrund aus Sand, sind durch Gezeiten, Strömungen und Stürme sich
ständig verändernde Sandbänke für das Brechen der Wellen
verantwortlich. Weitere wichtige Faktoren, die sich auf die
Eigenschaften von surfbaren Wellen auswirken, sind die Gezeiten, die
Richtung aus der die Wellen kommen, sowie die Wind- und
Wetterverhältnisse in den jeweiligen Jahreszeiten.

Die sogenannten \textit{Stormrider Guides} des \textit{Low Pressure}
Verlags sind seit langem die populärsten Reiseführer in der
Surfszene. Sie erfreuen sich dank der vielen hilfreichen Informationen
und Tipps einer sehr großen Beliebtheit. Die nach Kontinenten und
Ländern gegliederten Bücher enthalten Reiseinformationen über Land und
Leute, Kultur, Klima sowie Kartenausschnitte mit Beschreibungen zu den
Surfbedingungen, die an den jeweiligen Spots herrschen. Insbesondere
die detaillierten Informationen über die Eigenschaften der Wellen und
der Umgebung sind von großem Nutzen. Zum Beispiel wird beschrieben zu
welcher Gezeit bzw. Tide die Wellen an einem Spot am besten brechen,
wie stark die Meeresströmung ist oder ob die Brandung an einem
Sandstrand oder auf einem flachen Riff ist.

% 
{\sf \footnotesize
  \begin{tabular}{|p{2.5cm}p{0.7cm}p{0.7cm}|}
    \hline
    & & \\
    \multicolumn{3}{|l|}{\textbf{Hai Angriff Statistik}} \\
    \multicolumn{3}{|l|}{für Länder mit mehr als 10 Angriffen} \\
    & & \\
    & \textbf{Tota}l & \textbf{Fatal} \\
    \textbf{Europa} & \textbf{38} & \textbf{18} \\
    Italien & 14 & 4 \\
    \textbf{Afrika} & \textbf{255} & \textbf{67} \\
    Südafrika & 208 & 41 \\
    Mosambik & 11 & 3 \\
    \textbf{Indischer Ozean} & \textbf{62} & \textbf{27} \\
    Maskarenen & 21 & 12 \\
    Iran & 23 & 8 \\
    Indien & 10 & 4 \\
    \textbf{Ost Asien} & \textbf{89} & \textbf{38} \\
    Papua-Neuguinea & 36 & 15 \\
    Philippinen & 15 & 6 \\
    Japan & 19 & 12 \\
    \textbf{Australien \& Neuseeland} & \textbf{326} & \textbf{141} \\
    West Australien & 28 & 9 \\
    Süd Australien & 30 & 16 \\
    Victoria & 20 & 8 \\
    Tasmanien & 16 & 6 \\
    Neusüdwales & 123 & 62 \\
    Queensland & 101 & 47 \\
    \textbf{Pazifik} & \textbf{211} & \textbf{62} \\
    Marshallinseln & 12 & 0 \\
    Salomon-Inseln & 17 & 8 \\
    Fiji & 25 & 10 \\
    Hawaii & 104 & 19 \\
    \textbf{Nord Amerika} & \textbf{720} & \textbf{38} \\
    Oregon & 17 & 1 \\
    Kalifornien & 111 & 8 \\
    Texas & 30 & 3 \\
    Florida & 187 & 13 \\
    Südkarolina & 43 & 3 \\
    Nordkarolina & 24 & 3 \\
    New Jersey & 16 & 5 \\
    \textbf{Zentralamerika} & \textbf{118} & \textbf{50} \\
    Mexiko & 39 & 21 \\
    Panama & 16 & 9 \\
    \textbf{Südamerika} & \textbf{89} & \textbf{21} \\
    Brasilien & 81 & 20 \\
    & & \\
    \textbf{TOTAL} & \textbf{1909} & \textbf{456} \\
    & & \\
    \multicolumn{3}{|l|}{The International Shark Attack File,} \\
    \multicolumn{3}{|l|}{Florida Museum of Natural History,} \\
    \multicolumn{3}{|l|}{University of Florida.} \\
%    \multicolumn{3}{|l|}{{\tiny http://www.flmnh.ufl.edu/fish/Sharks/ISAF/ISAF.htm}} \\
    \hline
  \end{tabular}
}

%%% Local Variables:
%%% mode: latex
%%% TeX-master: "../community-plattform"
%%% End:


%%% Local Variables:
%%% mode: latex
%%% TeX-master: "../community-plattform"
%%% End:
