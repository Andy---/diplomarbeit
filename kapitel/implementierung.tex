\chapter{Implementierung der Web Applikation}
\section{Vorstellung der verwendeten Technologien}

\section{Behaviour Driven Development}

\textit{Behaviour Driven Development} ist eine Erweiterung des
\textit{Test Driven Development}, einer Methode aus dem Bereich der
\textit{Agilen Softwareentwicklung}. Ziel der Agilen
Softwareentwicklung ist es den klassischen Softwareentwicklungsprozess
schlanker und flexibler zu gestalten und sich mehr auf die zu
erreichenden Ziele und Anforderungen zu
fokussieren. Paarprogrammierung, ständige Refaktorisierungen des Codes
und testgetriebene Entwicklung sind einige der dabei verwendeten
Methoden \cite{wiki:agile}.

Behaviour Driven Development ist ein durch Dan North geprägter Begriff
und eine Erweiterung der Philosophie des Test Driven
Developments. Durch die Einführung eines gemeinsames Vokabluars soll
die Zusammenarbeit zwischen Programmieren und anderen
nicht-technischen Beteiligten an einem Softwareprojekt erhöht
werden. Das gemeinsame Vokabular wird dazu benutzt das Verhalten,
bzw. die Anforderungen der Software gemeinsam zu beschreiben. Diese
Anforderungen werden dann in Form von automatischen Tests vom
Programmierer implementiert. Wie beim Test Driven Development wird der
Test auch hier vor der Implementierung geschrieben. Bei dieser Art der
Software Entwicklung wird die API des zu entwickelnden Codes während
der Beschreibung und der Implementierung des Verhaltens in Form von
Tests erforscht.

Rspec ist ein Behaviour Driven Development Framework für Ruby, das
unter anderen auch von Dan North mitentwickelt wird. Das
\textit{Spec::Rails} Plugin bietet die nötige Infrastruktur für Ruby
on Rails Projekte. An die MVC-Architektur von Rails angelehnt bietet
es verschiedene Umgebungen um Model, View, Controller und Helper zu
spezifizieren.

\begin{verbatim}
A saved spot
- should have a name
- should have a secret flag
- should not be secret by default
- should not be valid without name
- should not be valid without user

A saved public spot
- should be public
- should not be secret
- should return the invitation status for the spot owner
- should return the invitation status an anonymous user
- should be visible to an anonymous user
- should be visible to the given user

A saved secret spot without a spot invitation
- should have the spot owner as inviter
- should have the spot owner as invitee
- should be secret
- should not be public

\end{verbatim}



\section{Architektur der Web Applikation}
\section{Caching Verfahren in Ruby on Rails}

%%% Local Variables:
%%% mode: latex
%%% TeX-master: "../community-plattform"
%%% End:
