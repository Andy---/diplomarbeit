
\chapter{Aufbau und Analyse der ETL Prozesse}
\section{Überblick der verwendeten Datenquellen}

\subsection{Wetter- und Wellendaten}

Heutige Wettervorhersagen basieren meist auf den Ergebnissen
\textit{numerischer Wettermodelle}, die von verschiedenen
Wetterdiensten erstellt und betrieben werden. Diese Modelle versuchen
den Zustand der Atmosphäre und deren Veränderung als mathematisches
Problem zu berschreiben. Eingabe für ein solches Modell ist der
Zustand der Atmosphäre in Form von physikalischen Größen, wie
z.B. Temperatur, Windstärke und Windrichtung. Die physikalischen
Beziehungen, die den Zustand der Atmosphäre verändern werden als
System partieller Differentialgleichungen modelliert.

% Verfahren der Numerik
% und der Einsatz von Supercomputern helfen dabei d


% formuliert. Zur Lösung des Problems werden Verfahren

% Dieses wird mit Verfahren der Numerik und dem Einsatz von
% Supercomputern näherungsweise gelöst. Das Ergebnis

% Es gibt eine Vielzahl dieser Modelle die
% von unterschiedlichen Wetterdiensten erstellt werden, und aufgrund der
% Anwendung verschiedener Verfahren auch erhebliche Abweichungen



% Diese Ergebnisse repräsentieren
% den Zustand der Atmosphäre für ein Vorhersagegebiet zu einem
% bestimmten Zeitpunkt in Form von physikalischen Größen, wie
% z.B. Temperatur, Windstärke und Windrichtung.



% Diese E



% Das Vorhersagegebiet wird dabei in Gitterzellen


%  als dreidimensionaler Raum in
% Abhängigkiet der Zeit betrachtet.

% in Gitterzellen aufgeteilt, um die für
% das Modell relevanten physikalischen Größen als Funktion der Zeit im
% dreidimensionalen Raum darstellen zu können.

% .System


%  in Form
% von partiellen Differentialgleichungen dar.


% Die Berechnung dieser
% Modelle ist sehr aufwendig, weshalb Verfahren aus dem Bereich der
% Numerik und Supercomputer verwendet werden um das Problem
% näherungsweise zu lösen.

% Der Zustand der Atmosphäre


% . Dieser
% dreidimensionale Raum wird durch die geographischen Koordinaten
% Latitude, Longitude sowie der Höhe über dem Meeresspiegel aufgespannt.

% Die Eckpunkte
% einer Gitterzelle representieren dabei die physikalischen Größen wie
% z.B. Temperatur, Windrichtung und -stärke, Wellenhöhe und
% -periode sind einige der






% , wie Temperatur,
% Luftdruck, Windrichtung, Windstärke, etc.

% Problem zu formulieren.

% , und durch die Lösung darzustellen.

% System partieller
% Differentialgleichungen darzustellen.


% zu einem gegeben Zeitpunkt durch die
% numerische Lösung


% Es gibt eine Vielzahl von
% Wettermodellen die von verschiedenen Wetterdiensten für bestimmte
% Regionen angeboten werden.


% Für die Berechnung dieser Modelle
% wird ein Gebiet oder Region in Gitterzellen aufgeteilt.


% Es gibt eine Vielzahl dieser
% Modelle,


% In einem solchen Modell wir das


% , die sind rechnergestützte Wettervorhersagen

% Die für die Wetter- und Wellenvorhersagen benötigten Daten werden von
% der US-amerikanischen Wetter- und Ozeanografiebehörde \textit{National
%   Oceanic and Atmospheric Administration (NOAA)} als Ergebnis
% numerisch gelöster Wettermodelle zur Verfügung gestellt.

\subsection{Photos \& Videos}
\section{Datenbank Design}
\subsection{Konzeptionelle Schema}
\subsection{Physisches Schema}
\section{Extraktion aus den Quellsystemen}
\section{Transformation der Daten}
\section{Laden der Daten}
\section{Verbesserungen}

%%% Local Variables:
%%% mode: latex
%%% TeX-master: "../community-plattform"
%%% End:
