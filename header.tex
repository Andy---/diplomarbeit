
\usepackage[usenames,dvipsnames]{color}
\definecolor{LightGray}{gray}{0.95}

\usepackage[utf8]{inputenc}

\usepackage[T1]{fontenc}
\usepackage[utf8]{inputenc}
\usepackage{lmodern}

\usepackage{longtable}

\usepackage{booktabs}
\renewcommand{\arraystretch}{1.4}

% Fuer eine Liste mit Abk.
\usepackage[intoc,german]{nomencl}
\renewcommand{\nomname}{Abkürzungsverzeichnis}
\setlength{\nomlabelwidth}{.20\hsize}
\renewcommand{\nomlabel}[1]{#1 \dotfill}
\setlength{\nomitemsep}{-\parsep}

\usepackage{bibgerm}
\usepackage{babel}
\usepackage{graphicx}
\usepackage{float}
\usepackage{subfigure}

\usepackage{url}
\usepackage[colorlinks=true,linkcolor=black,citecolor=black,pagecolor=black,urlcolor=black]{hyperref}

\usepackage{fancyhdr}
\pagestyle{headings}
\usepackage{titlesec}

\titleformat{\chapter}[display]
{\bfseries\Large}
{\filleft\MakeUppercase{\chaptertitlename} \Huge\thechapter}
{4ex}
{\titlerule
  \vspace{2ex}%
  \filright}
[\vspace{2ex}%
\titlerule]


% LISTINGS
\usepackage{listings}
\renewcommand{\lstlistingname}{Auflistung}
\lstset{
  basicstyle=\ttfamily\footnotesize,
  breaklines=true,
  captionpos=b,
  float=htbp,
  frame=single,
  showspaces=false,
  showstringspaces=false,
  numbers=left,
  numberstyle=\footnotesize,
  backgroundcolor=\color{LightGray}
}

%%% Local Variables:
%%% mode: latex
%%% TeX-master: "community-plattform"
%%% End:
